\documentclass{beamer}
  
% See this: http://texblog.net/latex-archive/uncategorized/beamer-warnings/
\let\Tiny=\tiny

\usepackage[
    compress,
    %minimal,
    %nonav,
    red,
    %gold,
    %blue,
    %numbers,
    %nologo,
    %nominilogo,
    minilogoleft,
    polyu,
    comp,
    forty,
    %seventyfive,
]  
{beamerthemeHongKong}
% add vertical space so footnotes do not run into navigation symbols (note: if you use more than one footnote, probably this will add an extra space)
\addtobeamertemplate{footnote}{}{\vspace{2ex}}
\usepackage{mathtools} 
\usepackage{booktabs}
\usepackage{multimedia}
\usepackage{relsize} 
\usepackage{transparent}
\usepackage{caption}
\captionsetup{font=small}
\captionsetup{labelformat=empty}

\title[Short title]{Long Title}
\author[Short name list for author(s)]{Full names for author(s)}
\institute[UW -- Madison, (indicate your department if desired)]{University of Wisconsin -- Madison \\ Department of [INSERT]}
\date{[INSERT DATE]} % or use \today for today's date

\begin{document}
  
% set footnote size to \tiny
\let\oldfootnotesize\footnotesize
\renewcommand*{\footnotesize}{\oldfootnotesize\tiny}

% set footnote macro for no marker
\newcommand\blfootnote[1]{%
  \begingroup
  \renewcommand\thefootnote{}\footnote{#1}%
  \addtocounter{footnote}{-1}%
  \endgroup
}

 

 

\frame{\titlepage}

\section*{Table of Contents} % example of section that will not be enumerated and will be excluded from \tableofcontents
\frame {
  \frametitle{\secname} % example to show each section (\secname) or subsection (\subsecname) can be referenced
  \tableofcontents
}

\AtBeginSubsection[] {
  \frame<handout:0> {
    \frametitle{Outline}
    \tableofcontents[current,currentsubsection]
  }
}

%====================================================================================================%

\section{Section 1}

%====================================================================================================%
%----------------------------------------------------------------------------------------------------%

\begin{frame}{insert title name} 
\centering insert text
\end{frame}

%----------------------------------------------------------------------------------------------------%
%====================================================================================================%

\subsection{Subsection 1.1}

%====================================================================================================%

%----------------------------------------------------------------------------------------------------%
  
\begin{frame}{Example of figure and block environments}
 Confinement requires toroidal magnetic fields with a poloidal twist
\begin{columns}
  \column{0.43\textwidth}
    \begin{figure} 
    \includegraphics[width=\textwidth]{graphics/tokamak_config}
    %\caption{Typical tokamak configuration} optional
    \end{figure}
  \column{0.57\textwidth}

    \begin{block}{Tokamak configuration}
      \begin{itemize}
        \item Toroidal field coils $\Rightarrow$ $B_{\phi}$
        \item Plasma current $\quad \,\,\Rightarrow$ $B_{\theta}$
        \item Poloidal field coils for shaping
        \item Transformer $\Rightarrow$ pulsed operation
      \end{itemize}
    \end{block}
  \end{columns}


\blfootnote{Image: obtained from \url{http://www.alternative-energy-action-now.com/tokamak-fusion-reactor.html}}

% \blfootnote places a footnote without a reference number

\end{frame}

%----------------------------------------------------------------------------------------------------%
%====================================================================================================%

\subsection{Subsection 1.2}

%====================================================================================================%
%----------------------------------------------------------------------------------------------------%

\begin{frame}{Example of table environment (tabular with booktabs)}
\begin{table}\centering
\begin{tabular}{@{}llclcl@{}}\toprule[2 pt]
& \multicolumn{1}{c}{NSTX}  && \multicolumn{1}{c}{ITER} \\
\midrule
\phantom{a}Parameters & &&  \\
\phantom{a}Major radius $R$ [m] 	&	$0.8-1.0$	 && 	$6.2$		\\
\phantom{a}Minor radius $a$ [m] 	&	$0.5-0.787$	 &&     $2$	        \\
\phantom{a}Aspect ratio	$R/a$           &       $1.27-1.6$	 && 	$3.1$		\\
\phantom{a}Magnetic field $B$ [T]       &  	$0.8 \sim 1.6$	 && 	$3.1$		\\
\phantom{a}Max plasma volume [$\text{m}^3$]	&14        	 && 	700		\\
\phantom{a}Max  $T_i, T_e$ [keV]	&	2.5, 4.1	 && 	30, 30		\\
\phantom{a}Max current $I$ [MA]         &	1.5        	 && 	15		\\
\phantom{a}Max power density [MW/$\text{m}^3$] &	1.1	         && 	0.7		\\
\phantom{a}Shot length [s] at $B_{max}$  &	1.5        	 && 	400		\\
\phantom{a}PFCs	                        &	CFC/Graphite	 && 	W, C and Be	\\
	                                &	Li coating	 && 			\\
\bottomrule[2 pt]
\end{tabular}
\label{tbl:tokamak_params_juxtaposition}
\end{table}

\tiny{Reproduced from MIT PSFC}

\end{frame}

%----------------------------------------------------------------------------------------------------%
%----------------------------------------------------------------------------------------------------%

\begin{frame}{Example of itemize environment}
\begin{center} \fbox{Theory} $\Leftrightarrow$ \fbox{Computation} $\Leftrightarrow$ \fbox{Experiment}\end{center}
Computer simulations are indispensible in contributing to the understanding of magnetically confined plasmas
\begin{itemize}
\item Numerical simulations help guide experiments by
  \begin{itemize}
    \item simulating and fine-tuning proposed experiments % $\Rightarrow$ reduces burden on experiments, and focuses both efforts and time
    \item producing results that have not yet been explored
    \item post-processing can simulate diagnostic measurements
    \item numerical solutions can aid design of devices (e.g. stellarators)
    \end{itemize}
\item Numerical simulations can access costly ventures (e.g. the \emph{burning plasma} experiment)
\item As more detailed and higher fidelity computational models are developed, the utility of computation increases and the feedback cycle becomes more efficient and useful.
\end{itemize}
\end{frame}

%----------------------------------------------------------------------------------------------------%
%====================================================================================================%

\section{Section 2}

%====================================================================================================%
%----------------------------------------------------------------------------------------------------%
\begin{frame}{Example of enumerate environment}
\begin{enumerate}
\item \textbf{Multiple species}: electrons, multiple ions, neutrals\\[0.5em]
\item \textbf{Time scales} span  $9 - 12$ orders of magnitude
\begin{itemize}
\item $\Omega_{ce}^{-1} \gtrsim 10^{-10}$ s : RF heating time scale \\[0.5em]
\item $\tau\phantom{_{ce}^{-1}} \lesssim 10^{2}\phantom{^{-0}}$ \,s : discharge time scale
\end{itemize}
\item \textbf{Spatial scales} span $4 - 6$ orders of magnitude
\begin{itemize}
\item $\lambda_{e,\nabla T} \gtrsim 10^{-5}$ m : $\nabla T$ length scale for electron conduction \\[0.5em]
\item $L\phantom{_{e,\nabla T}} \lesssim 40\phantom{^{-5}}$ m : connection length
\end{itemize}
\item \textbf{Velocity scales} span $\sim 7$ orders of magnitude
\begin{itemize}
% the author acknowledges the spacing/alignment here was done amateurishly, no excuses.
\item $v_n\phantom{_i} \quad \,\,\,\,\sim 0$ \qquad \,\,\,\,\,\,\,\,\,\, re-emitted neutrals adsorbed on wall \\[0.5em]
\item $v_{Ti} \quad \,\,\,\,\sim 36.12$ \quad \,\,\,\, m/s for $H^+$ thermal ions $T_i = 0.025$ eV\\[0.5em]
\item $v_{Te,SOL} \sim 10^6$\qquad\,\,\,\,\,\,\,m/s for $e^-$ in edge $T_e = 10 - 50$ eV \\ [0.5em]  
\item $v_{Te} \quad \,\,\,\sim 7.26\times 10^{7}$ m/s for $e^-$ at max $T_e \sim 30$ keV \\[0.5em]
\end{itemize}
\end{enumerate}

Fully dimensional simulation requires $10^{11}$ phase space cells with $10^8$ time steps over full discharge.

 

\end{frame}
 
%====================================================================================================%

\subsection{Subsection 2.1}

%====================================================================================================%
%----------------------------------------------------------------------------------------------------%

\begin{frame}{Example of math environment}
\vspace*{-2mm} % force write 2mm above beamer theme specifications for designated slide content area
The action of each operator = convection along characteristics!
\begin{eqnarray*}
\frac{\partial f_{\alpha}}{\partial t} + v\frac{\partial f_{\alpha}}{\partial x} + \frac{q_{\alpha}E}{m_{\alpha}}\frac{\partial f}{\partial v} & = & 0 \\
\frac{\partial f_{\alpha}}{\partial t} - \underbrace{\frac{1}{m_{\alpha}}\{H_T(v),f_{\alpha}\}}_{= \Lambda_xf_{\alpha}} -  \underbrace{\frac{1}{m_{\alpha}}\{H_V(x),f_{\alpha}\}}_{= \Lambda_vf_{\alpha}}  & = & 0 \\
\end{eqnarray*}

\vspace*{-3.5mm}The Vlasov equation is equivalent to

$$\frac{\partial f_{\alpha}}{\partial t} - \Lambda_xf_{\alpha} -  \Lambda_vf_{\alpha}   =  0 $$

And, these operators are used to solve the split problems
$$\frac{\partial f_{\alpha}}{\partial t} - v\frac{\partial f_{\alpha}}{\partial x}  =  0  \qquad\, \Rightarrow  f_{\alpha}(\tau ,x,v) = e^{\tau \Lambda_x}f_{\alpha}(0,x,v) = f_{\alpha}(0,x-v\tau , v)$$
$$\frac{\partial f_{\alpha}}{\partial t} - \frac{q_{\alpha}E}{m_{\alpha}}\frac{\partial f}{\partial v} =  0  \quad \Rightarrow  f_{\alpha}(\tau ,x,v) = e^{\tau \Lambda_v}f_{\alpha}(0,x,v) = f_{\alpha}(0,x,v-\tfrac{q_{\alpha}E}{m_{\alpha}}\tau )$$
 

\end{frame}

%----------------------------------------------------------------------------------------------------%
%====================================================================================================%

\subsection{Subsection 2.2}

%====================================================================================================%
%----------------------------------------------------------------------------------------------------%
%====================================================================================================%

\subsection*{Thanks!}

%====================================================================================================%

\begin{frame}{\subsecname}
\centering    \Huge{Thank you!}
\end{frame}


\end{document}


% below is a template slide that may be inserted between commented text dividers in order to rapidly
% create new slides while keeping the same organization in this tex file

%----------------------------------------------------------------------------------------------------%

\begin{frame}{Insert title}
Insert text
\end{frame}

%----------------------------------------------------------------------------------------------------%

% below is a new section or subsection template that can be inserted between commented text dividers
% and to still maintain the same organization in this tex file


%====================================================================================================%

\section{Section X}

%====================================================================================================%


